%% LyX 2.1.4 created this file.  For more info, see http://www.lyx.org/.
%% Do not edit unless you really know what you are doing.
\documentclass[12pt,english]{article}
\usepackage{helvet}
\usepackage[T1]{fontenc}
\usepackage[latin9]{inputenc}
\usepackage{geometry}
\geometry{verbose,tmargin=2cm,bmargin=2cm,lmargin=2cm,rmargin=2cm}
\setlength{\parindent}{0bp}
\usepackage{color}
\usepackage{babel}
\usepackage{listings}
\lstset{keywordstyle={\color{blue}}}
\begin{document}

\section*{Eveson: Sonification of CSRA Events}


\subsection*{Introduction: Ambient Sonification of Complex Continuous Status Data}

The \textsl{cognitive service robotics apartment} (\textbf{CSRA})
produces a large amount of event-driven data every second. As human
beings interacting with the apartment environment we could wish for
a solution to capture and understand this large volume of information
more easily and intuitively. The \textsl{Eveson }(short for \textsl{event
sonification}) project offers a way to bridge to gap between the myriad
of data generated by sensors and components and the impressive cognitive
abilities of human beings to perceive changes in the sounds sourrounding
us. By doing so we are capable of monitoring this complex data stream
for changes and therefore potential problems by informing the generation
of pleasant ambient soundscapes from the data.


\subsection*{The Cognitive Service Robotics Apartment}

The CSRA features a variety of rooms (e.g. kitchen, hallway, living
room) which are equipped with a large amount of sensors and components.
Up to this point the most relevant sensor data used for sonification
in the project comes from the motion sensors and powerconsumption
trackers installed in all rooms. Furthermore, steps on the kitchen
floor are sonified. 


\subsection*{Software and Methods used for Sonification of Events}

Control of events in the CSRA environment is achieved by the use of
\textsl{Robotics Service Bus} (\textbf{RSB}).

RSB is a message-oriented, event-driven middleware aiming at scalable
integration of robotics systems in diverse environment \footnote{https://code.cor-lab.org/projects/rsb}.
\\
It sends events taking place in the rooms of the apartment via an
\textsl{informer object} on a \textsl{scope} . The scope is given
as a string in the form \textsl{/home/kitchen. }Every event sent on
\textsl{/home/kitchen} can be received on\textsl{ /home} and every
event sent on\textsl{ /home} can be received at the root scope \textsl{/}.
Listening to scopes is implemented in the form of \textsl{listener
objects} that take a certain scope as an argument. An event-handler
can be attached to that listener to process the incomming events.\\
Eveson is using Java as a programming language and \textbf{\textsl{Jsyn}}
\footnote{http://www.softsynth.com/jsyn/} as a audio synthesis software
API to implement the audio functionality. 


\subsection*{How to use the Program}

At the current stage of the programm, two parameters can be supplied:\\
\\
1) -r \textsl{path\_to\_resource\_folder}\\
2) --theme \textsl{path\_to\_theme\_config}\\
\textsl{}\\
The resource folder contains all the sample files used by the program.
The\textsl{ theme-config} is a JSON object which can be found in the
\textsl{/cfg }folder of the project. It specifies:\\
1. \textsl{ID} \textsl{of the sensors}, i.e. MOTION\_SENSOR\_1\\
2. the respective \textsl{path to the sample files} for the sensor
and their scopes.

3. A \textsl{type} which describes in which ways the sample is played.\\
Furthermore it contains the relevant threshold values for the sonification
of the power consumption of the apartment (see features section for
more). \\
The resource folder is structured so that every sensor type has a
subfolder in the sample-folder (found at /\textsl{res}), i.e. ``Floor''
in \textsl{res/samples/Floor}. Each motion sensors has its own subfolder
where the all samples for that specific sensor are placed, i.e. \textsl{res/samples/Motionsensor/1.}


\subsection*{Features of the Program}

After starting the program on a device connected to the sound system
of the aparment, samples will be played on the speakers after the
corresponding event is triggered.

Currently Eveson supports the sonification of motion sensors, power
consumption and sounds for walking on the kitchen floor. The soundscape
is intended to mimic the sounds of nature to evoke a pleasant, familiar
feeling.\\
The folder structure of the programm permits adding new samples to
the specific sensor. It is achieved by just placing the sample file
in the corresponding folder and it will be played with a chance of
1 devided by the number of files in the folder.\\
\\
Moreover, by changing the theme-config file it is possible to add
more sensor types and specify additional sample-folder for them.\\
\\
An additional feature is the sonification of the power consuption
in the apartment. By listening and summing up all power consumption
events on \textsl{/home} we could measure the power consumption of
the entire apartment. For the sonification part we used a 3-stage
interpolation featuring different kinds of samples in the stages ranging
from slow wind to raoring thunder. These samples are stored in \textsl{res/samples/PowerConsumption/\{Normal,
High, Extreme\}.}


\subsection*{Future Work and Outlook}

There are two ways to further improve the program:\\
1. extend its current functionality\\
2. improve the soundscape using the existing set of features \\
\\
For 1. we could think of increasing the number of events included
in the sonification process. This encompasses more rooms as well as
adding more sensors to those rooms, e.g. ambient light and temperature
sensors. Also adding a non-sample based method to play sounds when
certain events are triggered is worth considering in the future. An
additional feature could be to implement that the sound samples are
only played on the speaker where the event originated.\\
For part 2. we need to further improve the soundscape created by the
current set of features by using more/better samples. This can be
achieved by consulting an expert in the field of audio perception
and design.
\end{document}
